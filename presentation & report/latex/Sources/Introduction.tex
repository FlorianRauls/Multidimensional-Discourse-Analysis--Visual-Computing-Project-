%%%%%%%%%%%%%%%%%%%%%%%%%%%%%%%%%%%%%%%%%%%%%%%%%%%%%%%%%%%%%%%%%%%%%%%%
\chapter{Introduction}

Microblogging services on social media platforms are rising in their popularity since their first appearance in the 2000s. Their importance is not only highlighted by the number of users which are using services like \emph{Twitter} daily, but also by being the center point of political discussions online. With this rise in importance, different research fields have been founded to analyze and understand behavioral patterns and information propagation inside them. Even though recent findings show that social networks are not representative of the overall population, ~\cite{mellon2017twitter} it is still important to understand the dynamics at play when working in fields like political science or communications. 

When short messages are posted online, many platforms provide some way of responding to any message for other users. A good example would be the \emph{retweet} feature on \emph{Twitter}. Any user can \emph{retweet} a message, therefore displaying the source to their followers, and in turn add their own message as a comment. This of course leads to chains of message, where often times discussions are held in the form of \emph{retweets} of \emph{retweets}. Even though one could easily follow a single chain of back and forth, any \emph{tweet} can have $n$ \emph{retweets}, which produces an ever-increasing complexity of threads to follow. 

We propose a visualization scheme for these discussion threads, where each message gets rendered at their temporal and spatial position, whilst message-answer-relations are indicated via pointers. Additional information like message \emph{sentiment} (tonal connotation) can be visualized through coloration, and distributions of any further dimensions can be shown in affixed plots right next to the rendered scene. Our dashboard also includes a graphical user interface, through which filters can be applied. Examples for filters could be maximum sentimental tone, usage of a so-called \emph{hashtag} or limitations in temporal occurrence. In combination, these tools enable users to narrow down messages to certain topics and any other desired specification. Complex research questions can be handled through these means, as one could e.g. ask \emph{"Does the tone of any given discussion get friendlier over time, when looking at the topic of school reforms?"}.