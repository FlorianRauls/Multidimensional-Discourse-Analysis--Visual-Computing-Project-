%%%%%%%%%%%%%%%%%%%%%%%%%%%%%%%%%%%%%%%%%%%%%%%%%%%%%%%%%%%%%%%%%%%%%%%%
\chapter{Conclusion}
We proposed a novel scheme for rendering diverse discussion threads on microblogging platforms, through which complex research questions, concerning the discussion, can be analyzed and answered. Our visualization can show temporal, spatial, sentimental and other dimensions at the same time, and reduce complexity for the user. Our combination of 3D-rendering of spatial and temporal data points with aggregated statistics and the option to introduce filters, allows for a very precise and in-depth answering of research questions, which can be proposed by users, who want to analyze specific aspects of conversations over microblogging sites. We also show a possible pipeline for preparing any data from a social media platform, for usage in our dashboard. Our experiments show, that this scheme can be used effectively to answer research questions about discourse online, while still minimizing the complexity in terms of visualization and user interface. Further research should try more complex questions on more complex data, to evaluate the possibilities further.  

Further research should look into ways of reducing the memory-complexity of our method, which is the main bottleneck right now, when trying to incorporate large amounts of data ($>3000$ data points) at once. Methods for streaming the data, instead of loading it into memory at once, could prove useful. Another task for further exploration could be the incorporation of additional visualization methods for different dimensions. Methods from fields like optical flow could be used to indicate changes in values over time and/or space very intuitively. Finally, improvements in the modularity of our tool would be highly welcome. Right now, it is focused on visualizing spatial data from the United States. A procedural approach for rendering maps on different scales on the fly, would make it so that discourse over international borders could be analyzed more easily. Additionally, looking into ways of tracking the flow of different topics in conversations could be worthwhile. One could for example filter for change of topics in conversational threads and correlating changes in e.g. sentiments or other variables.
To further possible insights into how to improve our methods, we strongly encourage the usage of our tools in fields like political science or communications, for analysis of online discourse.
%%%%%%%%%%%%%%%%%%%%%%%%%%%%%%%%%%%%%%%%%%%%%%%%%%%%%%%%%%%%%%%%%%%%%%%%