%%%%%%%%%%%%%%%%%%%%%%%%%%%%%%%%%%%%%%%%%%%%%%%%%%%%%%%%%%%%%%%%%%%%%%%%
\chapter{Related Work}
\label{sec:relatedWork}
%%%%%%%%%%%%%%%%%%%%%%%%%%%%%%%%%%%%%%%%%%%%%%%%%%%%%%%%%%%%%%%%%%%%%%%%

Since social media and online discourse are very wide topics, with lots of points of interest, a lot of research went into different forms of discourse analysis. For once topic modeling on specific points of discourse, like done by~\cite{TORNBERG2016132}, who used topic modeling and critical discourse analysis to investigate the portrayal of Muslim communities online,  is very popular. 

For methods of visualization~\cite{mckelvey2012visualizing} are closely related to our approach, as they used diffusion networks to show relations between different users on \emph{Twitter}. In contrast to our work, they present a more aggregated view on the discourse, focusing on meta-relations, instead of individual discussion threads. 

Another dashboard-like approach was chosen by ~\cite{SCHARL2016129} who devised an intelligence portal, which visualizes discourse held over social media and the news about the TV-series "Game of Thrones". While they also provide an interactive dashboard and visual analytic tools, they focus on covering more of the different textual dimensions and providing a multitude of different plots at once.