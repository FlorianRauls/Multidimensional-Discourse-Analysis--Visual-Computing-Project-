\begin{center}
\abstract{
  We propose a novel tool for visualizing the flow of discussions over temporal, spatial, sentimental and other custom dimensions simultaneously, by utilizing meta-data which is contained in most messages written on wide-spread microblogging services. For this we built a pipeline, which can automatically collect and process messages posted on the popular social media network \emph{Twitter}, developed a scheme for presenting the diverse range of data points in such a way, that an user can understand and filter them intuitively and evaluated our approach by analyzing discussions which are contained in our dataset. We furthermore demonstrate on how to best use our software, by describing an example research scenario and give instructions on how to solve it. Our tool can be used for detailed analysis of conversations on social media platforms. Our tool can be used to answer questions, which might be raised about discussions online and development of conversations.
We also discuss issues which arise from different visualization schemes and how to tackle them in the future. 
 }
\end{center}
%
\noindent 

